% Options for packages loaded elsewhere
\PassOptionsToPackage{unicode}{hyperref}
\PassOptionsToPackage{hyphens}{url}
\PassOptionsToPackage{dvipsnames,svgnames,x11names}{xcolor}
%
\documentclass[
  letterpaper,
  DIV=11,
  numbers=noendperiod]{scrartcl}

\usepackage{amsmath,amssymb}
\usepackage{iftex}
\ifPDFTeX
  \usepackage[T1]{fontenc}
  \usepackage[utf8]{inputenc}
  \usepackage{textcomp} % provide euro and other symbols
\else % if luatex or xetex
  \usepackage{unicode-math}
  \defaultfontfeatures{Scale=MatchLowercase}
  \defaultfontfeatures[\rmfamily]{Ligatures=TeX,Scale=1}
\fi
\usepackage{lmodern}
\ifPDFTeX\else  
    % xetex/luatex font selection
\fi
% Use upquote if available, for straight quotes in verbatim environments
\IfFileExists{upquote.sty}{\usepackage{upquote}}{}
\IfFileExists{microtype.sty}{% use microtype if available
  \usepackage[]{microtype}
  \UseMicrotypeSet[protrusion]{basicmath} % disable protrusion for tt fonts
}{}
\makeatletter
\@ifundefined{KOMAClassName}{% if non-KOMA class
  \IfFileExists{parskip.sty}{%
    \usepackage{parskip}
  }{% else
    \setlength{\parindent}{0pt}
    \setlength{\parskip}{6pt plus 2pt minus 1pt}}
}{% if KOMA class
  \KOMAoptions{parskip=half}}
\makeatother
\usepackage{xcolor}
\setlength{\emergencystretch}{3em} % prevent overfull lines
\setcounter{secnumdepth}{-\maxdimen} % remove section numbering
% Make \paragraph and \subparagraph free-standing
\ifx\paragraph\undefined\else
  \let\oldparagraph\paragraph
  \renewcommand{\paragraph}[1]{\oldparagraph{#1}\mbox{}}
\fi
\ifx\subparagraph\undefined\else
  \let\oldsubparagraph\subparagraph
  \renewcommand{\subparagraph}[1]{\oldsubparagraph{#1}\mbox{}}
\fi


\providecommand{\tightlist}{%
  \setlength{\itemsep}{0pt}\setlength{\parskip}{0pt}}\usepackage{longtable,booktabs,array}
\usepackage{calc} % for calculating minipage widths
% Correct order of tables after \paragraph or \subparagraph
\usepackage{etoolbox}
\makeatletter
\patchcmd\longtable{\par}{\if@noskipsec\mbox{}\fi\par}{}{}
\makeatother
% Allow footnotes in longtable head/foot
\IfFileExists{footnotehyper.sty}{\usepackage{footnotehyper}}{\usepackage{footnote}}
\makesavenoteenv{longtable}
\usepackage{graphicx}
\makeatletter
\def\maxwidth{\ifdim\Gin@nat@width>\linewidth\linewidth\else\Gin@nat@width\fi}
\def\maxheight{\ifdim\Gin@nat@height>\textheight\textheight\else\Gin@nat@height\fi}
\makeatother
% Scale images if necessary, so that they will not overflow the page
% margins by default, and it is still possible to overwrite the defaults
% using explicit options in \includegraphics[width, height, ...]{}
\setkeys{Gin}{width=\maxwidth,height=\maxheight,keepaspectratio}
% Set default figure placement to htbp
\makeatletter
\def\fps@figure{htbp}
\makeatother

\KOMAoption{captions}{tableheading}
\makeatletter
\@ifpackageloaded{caption}{}{\usepackage{caption}}
\AtBeginDocument{%
\ifdefined\contentsname
  \renewcommand*\contentsname{Table of contents}
\else
  \newcommand\contentsname{Table of contents}
\fi
\ifdefined\listfigurename
  \renewcommand*\listfigurename{List of Figures}
\else
  \newcommand\listfigurename{List of Figures}
\fi
\ifdefined\listtablename
  \renewcommand*\listtablename{List of Tables}
\else
  \newcommand\listtablename{List of Tables}
\fi
\ifdefined\figurename
  \renewcommand*\figurename{Figure}
\else
  \newcommand\figurename{Figure}
\fi
\ifdefined\tablename
  \renewcommand*\tablename{Table}
\else
  \newcommand\tablename{Table}
\fi
}
\@ifpackageloaded{float}{}{\usepackage{float}}
\floatstyle{ruled}
\@ifundefined{c@chapter}{\newfloat{codelisting}{h}{lop}}{\newfloat{codelisting}{h}{lop}[chapter]}
\floatname{codelisting}{Listing}
\newcommand*\listoflistings{\listof{codelisting}{List of Listings}}
\makeatother
\makeatletter
\makeatother
\makeatletter
\@ifpackageloaded{caption}{}{\usepackage{caption}}
\@ifpackageloaded{subcaption}{}{\usepackage{subcaption}}
\makeatother
\ifLuaTeX
  \usepackage{selnolig}  % disable illegal ligatures
\fi
\usepackage{bookmark}

\IfFileExists{xurl.sty}{\usepackage{xurl}}{} % add URL line breaks if available
\urlstyle{same} % disable monospaced font for URLs
\hypersetup{
  colorlinks=true,
  linkcolor={blue},
  filecolor={Maroon},
  citecolor={Blue},
  urlcolor={Blue},
  pdfcreator={LaTeX via pandoc}}

\author{}
\date{}

\begin{document}

\subsection{Curriculum vitae}\label{curriculum-vitae}

\subsubsection{Education}\label{education}

\begin{itemize}
\tightlist
\item
  University of Michigan (2021-Present)\\

  \begin{itemize}
  \tightlist
  \item
    \textbf{PhD Candidate}, Department of Ecology and Evolutionary
    Biology\\
  \item
    U-M Graduate Teacher Certificate, Center for Research on Learning \&
    Teaching (Completed)\\
  \item
    Complex Systems Graduate Certificate, Center for the Study of
    Complex Systems (In progress)\\
    \strut \\
  \end{itemize}
\item
  Indian Institute of Science (2016-2021)

  \begin{itemize}
  \tightlist
  \item
    Bachelor of Science (BS) + Master of Science (MS) in Biology
  \end{itemize}
\end{itemize}

\subsubsection{Teaching}\label{teaching}

EEB 429

Introduction to Statistical Model Building in R\\
Graduate Student Instructor, University of Michigan; Winter 2024

EEB 485

Population and Community Ecology (Graduate)\\
Graduate Student Instructor, University of Michigan; Fall 2022 \& Fall
2023

CMPLXSYS 391

Modeling Political Processes\\
Graduate Student Instructor, University of Michigan; Winter 2022

BIO 173

Introduction to Biology Lab (Undergraduate)\\
Graduate Student Instructor, University of Michigan; Fall 2021

\subsubsection{Publications \&
Conferences}\label{publications-conferences}

\textbf{Publications}

2023

\begin{itemize}
\tightlist
\item
  \textbf{Kumawat, B.}, Lalejini, A., Acosta, M. and Zaman, L., 2023.
  Fluctuating environments promote evolvability by shaping adaptive
  variation accessible to populations. \emph{bioRxiv}, pp.2023-01.
  \url{https://doi.org/10.1101/2023.01.04.520634}
\end{itemize}

2021

\begin{itemize}
\item
  \textbf{Kumawat, B.} and Zaman, L., 2021, July. Architecture of the
  Genotype-Phenotype Map and the Coevolution of Complexity. In
  \emph{Artificial Life Conference Proceedings 33} (Vol. 2021, No.~1,
  p.~66). MIT Press. \url{https://doi.org/10.1162/isal_a_00386}\\
\item
  \textbf{Kumawat, B.} and Bhat, R., 2021. An interplay of resource
  availability, population size and mutation rate potentiates the
  evolution of metabolic signaling. \emph{BMC Ecology and Evolution,
  21}, pp.1-15. \url{https://doi.org/10.1186/s12862-021-01782-0}
\end{itemize}

\textbf{Book Chapters}

2019

\begin{itemize}
\tightlist
\item
  D'Costa, J., Pujar, A., \textbf{Kumawat, B.}, Venkatesh, P., Ranjith,
  G., Sinha, V., Dubey, A.K., Narayan, H. \emph{Resistance: Tales from a
  Post-Antibiotic World. IISc Press}, 2019. ISBN-10: 8192570789.
\end{itemize}

\textbf{Conferences}

2024

\begin{itemize}
\item
  Complex Systems Summer School at the Santa Fe Institute, Santa Fe, New
  Mexico.
\item
  MAC-EPID Symposium on climate change and health: Microbial threats and
  microbial solutions. Talk on the directed evolution of evolvability
  for enhanced phage therapy.
\end{itemize}

2023

\begin{itemize}
\item
  EMBO Workshop: Predicting evolution, Heidelberg. Flash Talk and
  Poster: ``Localization on phenotypic boundaries enhances population
  evolvability''
\item
  GRC/S Molecular Mechanisms in Evolution, Easton. Poster:
  ``Localization on phenotypic boundaries enhances population
  evolvability''
\item
  EMBL Symposium: The organism and its environment, Heidelberg. Talk:
  ``Selective capture at phenotypic boundaries enhances population
  evolvability in a changing environment''
\end{itemize}

2022

\begin{itemize}
\tightlist
\item
  Complex Systems Advanced Academic Workshop at Center for the Study of
  Complex Systems, University of Michigan, Ann Arbor. Talk on evolution
  of evolvability.
\end{itemize}

2021

\begin{itemize}
\tightlist
\item
  Alife 2021, Prague, organised by the International Society for
  Artificial Life. Talk on the paper ``Architecture of the
  Genotype-Phenotype Map and the Coevolution of Complexity''
\end{itemize}

2020

\begin{itemize}
\tightlist
\item
  ALife 2020, Montreal, organised by the International Society for
  Artificial Life. Attended the meeting as a new member to Alife the
  community.
\end{itemize}

2019

\begin{itemize}
\item
  Indo-Swiss Meeting on Evolutionary Biology, CHG, Bangalore. Poster on
  ``Relatively disparate evolutionary dynamics of genomic and
  developmental features in unicellular and multicellular contexts''
\item
  Indo-Swiss Meeting on Evolutionary Biology, CHG, Bangalore. Poster on
  ``Utility functions with compounding returns lead to evolution of
  cooperativity in Multi-Armed Bandit networks''
\item
  Indian Society of Evolutionary Biologists (ISEB) Annual Conference,
  JNCASR, Bangalore. Poster on ``Investigating the evolution of
  developmental mechanisms in digital multicellular organisms''
\item
  PhageShift talk at the Center For BioSystems Science And Engineering
  symposium, Indian Institute of Science, Bangalore.
\end{itemize}

2018

\begin{itemize}
\tightlist
\item
  PhageShift talk (and poster) at the International Genetic Engineering
  Machine Competition (iGEM) Giant Jamboree, Boston, MA. Winner of a
  Gold Medal and Best Software Tool Nomination.
\end{itemize}

\subsubsection{Awards \& Honors}\label{awards-honors}

\textbf{Academic}

2019

\begin{itemize}
\tightlist
\item
  Best Poster Award, iSEB 2019 annual conference, JNCASR, Bangalore
\end{itemize}

2018

\begin{itemize}
\tightlist
\item
  Gold Medal and Best Software Tool Nomination (Team Leader), iGEM 2018,
  Boston, MA
\end{itemize}

2017

\begin{itemize}
\tightlist
\item
  Gold Medal and Best Hardware Nomination, iGEM 2017, Boston, MA
\end{itemize}

\textbf{Fellowships/Funding}

2023

Rackham Research Grant

Research Grant by the Rackham Graduate School at University of Michigan
to expand evolvability project to the wet lab.

2022-23

EEB Block Grant

Research Grant by Department of Ecology and Evolutionary Biology,
University of Michigan

2017

iBEC Grant

Indian Biological Engineering Competition. Awarded ∼14,000\$ by the
Department of Biotechnology, Govt. of India for iGEM 2017.

2014

KVPY Fellowship

National level competitive scholarship with stipend upto pre-PhD level
by DST, Govt. of India.

2012

NTSE Scholarship

National level competitive scholarship by NCERT, Govt. of India
(\textless0.1\% selection rate)

\subsubsection{Service \& Outreach}\label{service-outreach}

2022

\begin{itemize}
\tightlist
\item
  Feria de Ciencias Volunteer, Ann Arbor High School. SACNAS @
  University of Michigan
\end{itemize}

2018

\begin{itemize}
\tightlist
\item
  Public engagement presentations in local high schools to discuss
  antimicrobial resistance, Bangalore, India (as a part of iGEM 2018)
\item
  Undergraduate Synthetic Biology Workshop as iGEM 2018 Team Leader,
  Indian Institute of Science, Bangalore, India
\end{itemize}

\subsubsection{Coursework}\label{coursework}

\textbf{Graduate Level}

Biology

Molecular Basis of Ageing and Regeneration, Elements of Structural
Biology, Molecular Systems Biology, Principles of Genetic Engineering,
Bioinformatics, Spatial and Stochastic Dynamics in Biology, Quantitative
Ecology, Population and Community Ecology, Evolutionary Biology

Physics

Condensed Matter Physics - I, Statistical Mechanics, Computational
Physics, Information Theory, Theory of Social and Technological Networks

Other

Game Theory, Numerical Solutions of Differential Equations, Dynamical
Systems Theory, History of Complex Systems

\textbf{Undergraduate}

Biology

Molecular Biology, Developmental Biology, Physiology, Biochemistry

Physics

Intermediate Thermal Physics and Physics of Materials, Materials
Thermodynamics

Other

Algorithms and Programming, Intro. to Electrical and Electronics
Engineering, Probability and Statistics

\subsubsection{Technical Skills}\label{technical-skills}

Programming

C, C++, Python, Rust, Mathematica, Bash, Embedded C

Design

Inkscape, GIMP, R \& ggplot, basic OpenSCAD, Digital Electronics, 3D
Printing, CNC Milling, Laser Cutting

Lab

Microbiology, Molecular Biology, Bacteriophage work, Basic Biochemistry,
and Design of synthetic gene circuits (Model Systems: Escherichia coli,
T4 Bacteriophage, Dictyostelium discoideum)



\end{document}
